% Weilei Zeng
% A circuit to measure XXX XXX in shor's code
% There are three versions of this circuit 1>standard one 2>simplified one 3>more simplified
% Here shows 1>the standard one
% 
%   qubit	   q0
%   qubit	   q1
%   qubit	   q2
%   qubit	   q3
%   qubit	   q4
%   qubit	   q5
%   qubit	   A0,0
% 
%   H	   q0
%   H	   q1
%   H	   q2
%   H	   q3
%   H	   q4
%   H	   q5
%   cnot	   q0,A0
%   cnot	   q1,A0
%   cnot	   q2,A0
%   cnot	   q3,A0
%   cnot	   q4,A0
%   cnot	   q5,A0
%   nop	   q0
%   nop	   q0
%   nop	   q0
%   nop	   q0
%   nop	   q0
%   nop	   q1
%   nop	   q1
%   nop	   q1
%   nop	   q1
%   nop	   q2
%   nop	   q2
%   nop	   q2
%   nop	   q3
%   nop	   q3
%   nop	   q4
%   H	   q0
%   H	   q1
%   H	   q2
%   H	   q3
%   H	   q4
%   H	   q5
%   measure  A0
%  Time 01:
%    Gate 00 H(q0)
%    Gate 01 H(q1)
%    Gate 02 H(q2)
%    Gate 03 H(q3)
%    Gate 04 H(q4)
%    Gate 05 H(q5)
%  Time 02:
%    Gate 06 cnot(q0,A0)
%  Time 03:
%    Gate 07 cnot(q1,A0)
%    Gate 12 nop(q0)
%  Time 04:
%    Gate 08 cnot(q2,A0)
%    Gate 13 nop(q0)
%    Gate 17 nop(q1)
%  Time 05:
%    Gate 09 cnot(q3,A0)
%    Gate 14 nop(q0)
%    Gate 18 nop(q1)
%    Gate 21 nop(q2)
%  Time 06:
%    Gate 10 cnot(q4,A0)
%    Gate 15 nop(q0)
%    Gate 19 nop(q1)
%    Gate 22 nop(q2)
%    Gate 24 nop(q3)
%  Time 07:
%    Gate 11 cnot(q5,A0)
%    Gate 16 nop(q0)
%    Gate 20 nop(q1)
%    Gate 23 nop(q2)
%    Gate 25 nop(q3)
%    Gate 26 nop(q4)
%  Time 08:
%    Gate 27 H(q0)
%    Gate 28 H(q1)
%    Gate 29 H(q2)
%    Gate 30 H(q3)
%    Gate 31 H(q4)
%    Gate 32 H(q5)
%    Gate 33 measure(A0)

% Qubit circuit matrix:
%
% q0: gAxA, gBxA, gCxA, gDxA, gExA, gFxA, gGxA, gHxA, n  
% q1: gAxB, n  , gCxB, gDxB, gExB, gFxB, gGxB, gHxB, n  
% q2: gAxC, n  , n  , gDxC, gExC, gFxC, gGxC, gHxC, n  
% q3: gAxD, n  , n  , n  , gExD, gFxD, gGxD, gHxD, n  
% q4: gAxE, n  , n  , n  , n  , gFxE, gGxE, gHxE, n  
% q5: gAxF, n  , n  , n  , n  , n  , gGxF, gHxF, n  
% A0: n  , gBxG, gCxG, gDxG, gExG, gFxG, gGxG, gHxG, N  

\documentclass[11pt]{article}
\input{xyqcirc.tex}

% definitions for the circuit elements

\def\gAxA{\op{H}\w\A{gAxA}}
\def\gAxB{\op{H}\w\A{gAxB}}
\def\gAxC{\op{H}\w\A{gAxC}}
\def\gAxD{\op{H}\w\A{gAxD}}
\def\gAxE{\op{H}\w\A{gAxE}}
\def\gAxF{\op{H}\w\A{gAxF}}
\def\gBxA{\b\w\A{gBxA}}
\def\gBxG{\o\w\A{gBxG}}
\def\gCxB{\b\w\A{gCxB}}
\def\gCxG{\o\w\A{gCxG}}
\def\gDxC{\b\w\A{gDxC}}
\def\gDxG{\o\w\A{gDxG}}
\def\gExD{\b\w\A{gExD}}
\def\gExG{\o\w\A{gExG}}
\def\gFxE{\b\w\A{gFxE}}
\def\gFxG{\o\w\A{gFxG}}
\def\gGxF{\b\w\A{gGxF}}
\def\gGxG{\o\w\A{gGxG}}
\def\gCxA{*-{}\w\A{gCxA}}
\def\gDxA{*-{}\w\A{gDxA}}
\def\gExA{*-{}\w\A{gExA}}
\def\gFxA{*-{}\w\A{gFxA}}
\def\gGxA{*-{}\w\A{gGxA}}
\def\gDxB{*-{}\w\A{gDxB}}
\def\gExB{*-{}\w\A{gExB}}
\def\gFxB{*-{}\w\A{gFxB}}
\def\gGxB{*-{}\w\A{gGxB}}
\def\gExC{*-{}\w\A{gExC}}
\def\gFxC{*-{}\w\A{gFxC}}
\def\gGxC{*-{}\w\A{gGxC}}
\def\gFxD{*-{}\w\A{gFxD}}
\def\gGxD{*-{}\w\A{gGxD}}
\def\gGxE{*-{}\w\A{gGxE}}
\def\gHxA{\op{H}\w\A{gHxA}}
\def\gHxB{\op{H}\w\A{gHxB}}
\def\gHxC{\op{H}\w\A{gHxC}}
\def\gHxD{\op{H}\w\A{gHxD}}
\def\gHxE{\op{H}\w\A{gHxE}}
\def\gHxF{\op{H}\w\A{gHxF}}
\def\gHxG{\meter\w\A{gHxG}}

% definitions for bit labels and initial states

\def\bA{ \q{q_{0}}}
\def\bB{ \q{q_{1}}}
\def\bC{ \q{q_{2}}}
\def\bD{ \q{q_{3}}}
\def\bE{ \q{q_{4}}}
\def\bF{ \q{q_{5}}}
\def\bG{\qv{A_{0}}{0}}

% The quantum circuit as an xymatrix

\xymatrix@R=5pt@C=10pt{
    \bA & \gAxA &\gBxA &\gCxA &\gDxA &\gExA &\gFxA &\gGxA &\gHxA &\n  
\\  \bB & \gAxB &\n   &\gCxB &\gDxB &\gExB &\gFxB &\gGxB &\gHxB &\n  
\\  \bC & \gAxC &\n   &\n   &\gDxC &\gExC &\gFxC &\gGxC &\gHxC &\n  
\\  \bD & \gAxD &\n   &\n   &\n   &\gExD &\gFxD &\gGxD &\gHxD &\n  
\\  \bE & \gAxE &\n   &\n   &\n   &\n   &\gFxE &\gGxE &\gHxE &\n  
\\  \bF & \gAxF &\n   &\n   &\n   &\n   &\n   &\gGxF &\gHxF &\n  
\\  \bG & \n   &\gBxG &\gCxG &\gDxG &\gExG &\gFxG &\gGxG &\gHxG &\N  
%
% Vertical lines and other post-xymatrix latex
%
\ar@{-}"gBxG";"gBxA"
\ar@{-}"gCxG";"gCxB"
\ar@{-}"gDxG";"gDxC"
\ar@{-}"gExG";"gExD"
\ar@{-}"gFxG";"gFxE"
\ar@{-}"gGxG";"gGxF"
}

\end{document}
