% Weilei Zeng
% A circuit to measure XXX XXX in shor's code
% There are three versions of this circuit 1>standard one 2>simplified one 3>more simplified
% Here shows 3>more simplified one. This one is optimized such that the time interval is minimal
% 
%   qubit	   q0
%   qubit	   q1
%   qubit	   q2
%   qubit	   A0,0
%   qubit	   A1,0
%   qubit	   q3
%   qubit	   q4
%   qubit	   q5
% 
% 
%   H	   A0
%   H	   A1
%   cnot	   A0,q0
%   cnot	   A0,q1
%   cnot	   A0,q2
%   cnot	   A1,q3
%   cnot	   A1,q4
%   cnot	   A1,q5
%   cnot	   A0,A1
%   H	   A0
%   H	   A1
%   
%   measure  A0
%  Time 01:
%    Gate 00 H(A0)
%    Gate 01 H(A1)
%  Time 02:
%    Gate 02 cnot(A0,q0)
%    Gate 05 cnot(A1,q3)
%  Time 03:
%    Gate 03 cnot(A0,q1)
%    Gate 06 cnot(A1,q4)
%  Time 04:
%    Gate 04 cnot(A0,q2)
%    Gate 07 cnot(A1,q5)
%  Time 05:
%    Gate 08 cnot(A0,A1)
%  Time 06:
%    Gate 09 H(A0)
%    Gate 10 H(A1)
%  Time 07:
%    Gate 11 measure(A0)

% Qubit circuit matrix:
%
% q0: n  , gBxA, n  , n  , n  , n  , n  , n  
% q1: n  , n  , gCxB, n  , n  , n  , n  , n  
% q2: n  , n  , n  , gDxC, n  , n  , n  , n  
% A0: gAxD, gBxD, gCxD, gDxD, gExD, gFxD, gGxD, N  
% A1: gAxE, gBxE, gCxE, gDxE, gExE, gFxE, n  , n  
% q3: n  , gBxF, n  , n  , n  , n  , n  , n  
% q4: n  , n  , gCxG, n  , n  , n  , n  , n  
% q5: n  , n  , n  , gDxH, n  , n  , n  , n  

\documentclass[11pt]{article}
\input{xyqcirc.tex}

% definitions for the circuit elements

\def\gAxD{\op{H}\w\A{gAxD}}
\def\gAxE{\op{H}\w\A{gAxE}}
\def\gBxD{\b\w\A{gBxD}}
\def\gBxA{\o\w\A{gBxA}}
\def\gCxD{\b\w\A{gCxD}}
\def\gCxB{\o\w\A{gCxB}}
\def\gDxD{\b\w\A{gDxD}}
\def\gDxC{\o\w\A{gDxC}}
\def\gBxE{\b\w\A{gBxE}}
\def\gBxF{\o\w\A{gBxF}}
\def\gCxE{\b\w\A{gCxE}}
\def\gCxG{\o\w\A{gCxG}}
\def\gDxE{\b\w\A{gDxE}}
\def\gDxH{\o\w\A{gDxH}}
\def\gExD{\b\w\A{gExD}}
\def\gExE{\o\w\A{gExE}}
\def\gFxD{\op{H}\w\A{gFxD}}
\def\gFxE{\op{H}\w\A{gFxE}}
\def\gGxD{\meter\w\A{gGxD}}

% definitions for bit labels and initial states

\def\bA{ \q{q_{0}}}
\def\bB{ \q{q_{1}}}
\def\bC{ \q{q_{2}}}
\def\bD{\qv{A_{0}}{0}}
\def\bE{\qv{A_{1}}{0}}
\def\bF{ \q{q_{3}}}
\def\bG{ \q{q_{4}}}
\def\bH{ \q{q_{5}}}

% The quantum circuit as an xymatrix

\xymatrix@R=5pt@C=10pt{
    \bA & \n   &\gBxA &\n   &\n   &\n   &\n   &\n   &\n  
\\  \bB & \n   &\n   &\gCxB &\n   &\n   &\n   &\n   &\n  
\\  \bC & \n   &\n   &\n   &\gDxC &\n   &\n   &\n   &\n  
\\  \bD & \gAxD &\gBxD &\gCxD &\gDxD &\gExD &\gFxD &\gGxD &\N  
\\  \bE & \gAxE &\gBxE &\gCxE &\gDxE &\gExE &\gFxE &\n   &\n  
\\  \bF & \n   &\gBxF &\n   &\n   &\n   &\n   &\n   &\n  
\\  \bG & \n   &\n   &\gCxG &\n   &\n   &\n   &\n   &\n  
\\  \bH & \n   &\n   &\n   &\gDxH &\n   &\n   &\n   &\n  
%
% Vertical lines and other post-xymatrix latex
%
\ar@{-}"gBxA";"gBxD"
\ar@{-}"gCxB";"gCxD"
\ar@{-}"gDxC";"gDxD"
\ar@{-}"gBxF";"gBxE"
\ar@{-}"gCxG";"gCxE"
\ar@{-}"gDxH";"gDxE"
\ar@{-}"gExE";"gExD"
}

\end{document}
