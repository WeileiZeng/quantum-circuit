%Weilei Zeng
%measure Y_0
% 
% 	 qubit	q0
% 	 qubit	A0,0
% 	 
% 	 cnot	q0,A0
% 	 h	q0
% 	 cnot	q0,A0
% 	 h	q0
% 
% 	 measure	A0
%  Time 01:
%    Gate 00 cnot(q0,A0)
%  Time 02:
%    Gate 01 h(q0)
%  Time 03:
%    Gate 02 cnot(q0,A0)
%  Time 04:
%    Gate 03 h(q0)
%    Gate 04 measure(A0)

% Qubit circuit matrix:
%
% q0: gAxA, gBxA, gCxA, gDxA, n  
% A0: gAxB, n  , gCxB, gDxB, N  

\documentclass[11pt]{article}
\input{xyqcirc.tex}

% definitions for the circuit elements

\def\gAxA{\b\w\A{gAxA}}
\def\gAxB{\o\w\A{gAxB}}
\def\gBxA{\op{H}\w\A{gBxA}}
\def\gCxA{\b\w\A{gCxA}}
\def\gCxB{\o\w\A{gCxB}}
\def\gDxA{\op{H}\w\A{gDxA}}
\def\gDxB{\meter\w\A{gDxB}}

% definitions for bit labels and initial states

\def\bA{ \q{q_{0}}}
\def\bB{\qv{A_{0}}{0}}

% The quantum circuit as an xymatrix

\xymatrix@R=5pt@C=10pt{
    \bA & \gAxA &\gBxA &\gCxA &\gDxA &\n  
\\  \bB & \gAxB &\n   &\gCxB &\gDxB &\N  
%
% Vertical lines and other post-xymatrix latex
%
\ar@{-}"gAxB";"gAxA"
\ar@{-}"gCxB";"gCxA"
}

\end{document}
