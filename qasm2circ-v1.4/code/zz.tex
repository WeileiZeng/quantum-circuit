% Weilei Zeng
% A simple circuit to measure Z_0Z_1
% 
%   qubit	   q0
%   qubit	   q1
%   qubit	   A0,0
% 
%   cnot	   q0,A0
%   cnot	   q1,A0
%   measure  A0
%  Time 01:
%    Gate 00 cnot(q0,A0)
%  Time 02:
%    Gate 01 cnot(q1,A0)
%  Time 03:
%    Gate 02 measure(A0)

% Qubit circuit matrix:
%
% q0: gAxA, n  , n  , n  
% q1: n  , gBxB, n  , n  
% A0: gAxC, gBxC, gCxC, N  

\documentclass[11pt]{article}
\input{xyqcirc.tex}

% definitions for the circuit elements

\def\gAxA{\b\w\A{gAxA}}
\def\gAxC{\o\w\A{gAxC}}
\def\gBxB{\b\w\A{gBxB}}
\def\gBxC{\o\w\A{gBxC}}
\def\gCxC{\meter\w\A{gCxC}}

% definitions for bit labels and initial states

\def\bA{ \q{q_{0}}}
\def\bB{ \q{q_{1}}}
\def\bC{\qv{A_{0}}{0}}

% The quantum circuit as an xymatrix

\xymatrix@R=5pt@C=10pt{
    \bA & \gAxA &\n   &\n   &\n  
\\  \bB & \n   &\gBxB &\n   &\n  
\\  \bC & \gAxC &\gBxC &\gCxC &\N  
%
% Vertical lines and other post-xymatrix latex
%
\ar@{-}"gAxC";"gAxA"
\ar@{-}"gBxC";"gBxB"
}

\end{document}
